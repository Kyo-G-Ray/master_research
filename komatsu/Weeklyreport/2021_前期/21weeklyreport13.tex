\documentclass[a4j,twocolumn]{ltjarticle}

%ページ設定
\setlength{\topmargin}{-10mm}
\setlength{\oddsidemargin}{-5mm}
\setlength{\evensidemargin}{-5mm}

\setlength{\textheight}{230mm}
\setlength{\textwidth}{170mm}
\setlength{\columnsep}{10mm}%コラムとコラムの間.よってコラムの横幅は(texwidth-columnsep)/2
\setlength{\footskip}{15mm}
%パッケージ
\usepackage{graphicx,color}
\usepackage{bm}
\usepackage{amsmath}
\usepackage{amsfonts}

%数学関数
\newcommand{\B}[1]{\bm #1} 
\newcommand{\s}[2]{{#1}\cdot{#2}} 
\newcommand{\R}{\mathbb{R}}
\newcommand{\M}{\mathbb{M}}
\newcommand{\df}[2]{\displaystyle{\frac{#1}{#2}}}
\newcommand{\qed}{
\begin{flushright}
\vspace{-8mm}
$\Box$
\end{flushright}
}

%ミニセクション
\newcommand{\minisection}[2]{\flushleft{\bf (#1)#2}\\ \ \ \ }

%図,表の名前変更
\renewcommand{\figurename}{Fig.}
\renewcommand{\tablename}{Tab.}

% refの拡張
\newcommand{\reffig}[1]{\figurename \ref{#1}}
\newcommand{\refeq}[1]{式(\ref{#1})}
\newcommand{\reftab}[1]{\tablename \ref{#1}}

% 図
\newcommand{\includefigure}[3]{
\begin{center}
\includegraphics[width=80mm]{#1}
\caption{#2}
\label{#3}
\end{center}
}
% 図 widthを変更したいときはこっち
\newcommand{\includefigurewidth}[4]{
\begin{center}
\includegraphics[width=#2]{#1}
\caption{#3}
\label{#4}
\end{center}
}

\usepackage{setspace} % setspaceパッケージのインクルード
\usepackage{enumitem}
\usepackage{graphicx}
\usepackage{amsmath}
%「Weekly Report」 
\newcommand{\Weekly}[5]{
\twocolumn[
 \begin{center}
  \bf
 第 #1 回 Weekly Report\\
 \huge
電気使用量予測のための深層学習手法\\

 \end{center}
 \begin{flushright}
  #2 月\ \ \  #3 日 \ \ \ #4 \\\
  #5
 \end{flushright}
]
}
%\setstretch{0.5} % ページ全体の行間を設定

\begin{document}

\Weekly{13}{7}{13}{(火)}{\ 小松 大起}
\section{はじめに}
\subsection{研究背景}

電力発電においてコストは我々消費者だけでなく, 
2011 年 3 月 11 日に発生した東日本大震災以降, 原子力発電は安全性に問題があるとされ 2021 年 5 月現在では停止中の 2 基も合わせて合計 9 基の原子力発電所が稼働するのみとなっている.

近年, 人工知能に関する多くの分野で情報処理技術として知的処理技術の一つである深層学習が用いられている. 
深層学習自体の歴史は古く,
1958 年には深層学習の構成要素の 1 つであるパーセプトロンが考案された.[1]
また,
深層学習とは,
ニューロンの層が多段に組み上げられたニューラルネットーワークのことを指す.[2]
深層学習が用いられる分野としては, 人物の行動認識や表情認識[3]に挙げられるような画像処理に関わるものや, 話し言葉や書き言葉などの我々が普段使うような自然言語を対象として, それらの言葉が持つ意味を解析する自然言語処理や, 株価や電気使用量などの予測にも用いられる. 
また, 近頃盛んに行われている自動運転にも深層学習の技術が取り入れられている. []

\subsection{研究目的}

電気使用量は人間の認知は時間経過による視覚世界の変化の予測が可能である. 近年では実際に予測動画を作る研究も行われてきている.[2]電力使用量を予測することによる, 電気料金の予測が可能になると考えられる. 本研究では, 電力使用量を主データとし, 天気や気温が与えうる電力使用量の変化を考慮した電気使用料の予測を行うことを目的とする. 

\section{深層学習モデル}
\subsection{RNN}
\subsection{LSTM}

\section{活性化関数}
活性化関数とは, ニューロン間の移動に伴い入力値を別の数値に変換して出力するための関数のことである.
\subsection{ステップ関数}
ステップ関数は, 入力が 0 未満の場合には常に出力値が 0 となり, 0 以上の場合には常に出力値が 1 となるような関数を指す. ステップ関数は, パーセプトロンから用いられている関数であり入力 0 を起点として階段状のグラフを示す. この起点を閾値と呼ぶ. 入力を $x$ として $f(x)$ を出力とすると数式は以下の式で表される.
\begin{equation}
f(x)= \begin{cases}
0, & (x < 0)\\
1, & (x \geq 1)
\end{cases}
\end{equation}

\section{事前実験}
\subsection{用いるモデル構造}
本実験では, RNN 及び LSTM を用いて電力使用量の予測を行う. また, 予測に用いるデータは電力使用量のみを用いる. 本実験では, 予測結果は全て t+1 ステップ後の結果を表している. 
\subsection{用いるデータ}
本実験で用いるデータは, 東京電力パワーグリッド株式会社が提供している 2016 年 4 月から 2020 年 12 月までの電力使用量のデータであり, 年度, 日にち, 1 時間ごとの電力使用量(万 Kw)の3 要素が csv ファイルで提供されている. 2016 年は 4 月からのデータのため, 6600rows * 3columns, 2017 年から 2019 年は 8760rows * 3columns, 2020 年は閏年であり通常よりも1日分多いため 24 列多い 8784rows * 3columns のデータである. また, 用いるデータの一例を図 0 に示す. これらのデータの日にちと時間を結合させて日時のデータとする. また, その際に日付と時間が並んでいるだけの文字列であるので, Python のデータ解析用ライブラリである pandas を用いて文字列を日付データに変換する. その例を図 0 に示す. また, データを可視化してグラフにしたものの例として 2019 年, 2020 年のグラフ及び,
2020 年 4 月の 5 日から 11 日, 12 日から 18 日, 19 日から 25 日の 3 週分のグラフをそれぞれ図に示す. 
2019 年, 2020 年のグラフからわかることは夏と冬に電力使用量のピークを迎え, 春と秋に使用量は減っていることがわかる. 時間帯によっても使用料の増減があることが日のグラフを見ることでわかる. 深夜から朝にかけて段々と使用料が増えていき, 人々が活動を行なっているであろう 9 時ごろから 18 時ごろまで使用量がピークで, そこから使用量が段々と下がっていることがわかる.
また, 電力使用量に関しては全てのデータの平均でそれぞれのデータを割って正規化を行なったものを入力データとして扱う.
このデータの可視化から全体を通して大まかに傾向があり, それにしたがっていることがわかるが所々傾向に沿っていないデータがあるが, それらは天気, 降水量, 気温の変化によってもたらされることがわかる.

\begin{figure}[hb]
\centering
\includegraphics[scale=0.5]{exe_csv.pdf}
 \caption{用いる電力使用量データ例}
\end{figure}

\begin{figure*}[hb]
\centering
\includegraphics[scale=0.5]{2019_W.png}
 \caption{2019 年の電力使用量のグラフ}
\end{figure*}

\begin{figure*}[hb]
\centering
\includegraphics[scale=0.5]{2020_W.png}
 \caption{2020 年の電力使用量のグラフ}
\end{figure*}

%% \begin{figure*}[hb]
%% \centering
%% \includegraphics[scale=0.5]{2020_week1.png}
%%  \caption{2020 年 4 月 5 日から 4 月 11 日の電力使用量の比較}
%% \end{figure*}
%% \begin{figure*}[hb]
%% \centering
%% \includegraphics[scale=0.5]{2020_week2.png}
%%  \caption{2020 年 4 月 12 日から 4 月 18 日の電力使用量の比較}
%% \end{figure*}
%% \begin{figure*}[hb]
%% \centering
%% \includegraphics[scale=0.5]{2020_week1.png}
%%  \caption{2020 年 4 月 19 日から 4 月 25 日の電力使用量の比較}
%% \end{figure*}

%% \begin{figure*}[hb]
%% \centering
%% \includegraphics[scale=0.5]{2020_week1.png}
%%  \caption{2020 年 4 月 5 日から 4 月 11 日の電力使用量の比較}
%% \end{figure*}
%% \begin{figure*}[hb]
%% \centering
%% \includegraphics[scale=0.5]{2020_week2.png}
%%  \caption{2020 年 4 月 12 日から 4 月 18 日の電力使用量の比較}
%% \end{figure*}
%% \begin{figure*}[hb]
%% \centering
%% \includegraphics[scale=0.5]{2020_week3.png}
%%  \caption{2020 年 4 月 19 日から 4 月 25 日の電力使用量の比較}
%% \end{figure*}

\subsubsection{RNN}
keras の simpleRNN というモデルを用いて訓練データと試験データを 8:2 に分けて学習データの作成を行った. 中間層 1 層, 隠れニューロン数は 100 とした. また、その時の予測の結果を図 0 に示す.

%% \begin{figure*}[ht]
%% \begin{center}
%% \includegraphics[scale=0.6]{rnn_pred_month.pdf}
%% \end{center}
%% \vspace{-80mm}
%% \caption{RNN 訓練データに対する予測結果}
%% \end{figure*}

%% \begin{figure*}[b]
%% \begin{center}
%% \includegraphics[scale=0.6]{lstm_pred_month.pdf}
%% \end{center}
%% \vspace{-80mm}
%% \caption{LSTM 訓練データに対する予測結果}
%% \end{figure*}

\subsubsection{LSTM}
RNN と同条件で予測を行った結果を以下の図に示す.

%% \begin{figure*}[ht]
%% \begin{center}
%% \includegraphics[scale=0.55]{rnn_pred_day.pdf}
%% \vspace{-75mm}
%% \caption{RNN テストデータに対する予測結果}
%% \end{center}
%% \end{figure*}

%% \begin{figure*}[phb]
%% \begin{center}
%% \includegraphics[scale=0.55]{lstm_pred_day.pdf}
%% \vspace{-75mm}
%% \caption{LSTM テストデータに対する予測結果}
%% \end{center}
%% \end{figure*}

\subsection{予測結果}
それぞれのモデル構造における Train loss 及び, Test loss を以下の表に示す.
結果から, 電気使用量のみを学習させた場合にも大まかな予測ができていることがわかる.


\section{用いる追加データ}
本研究では, 電力使用力の予測を天気, 気温などの外的要因から行うことを目的とする.
国土交通省の気象庁がホームページで公開している気温, 天気の情報を用いる.
また, そのデータの例を図 0 に示す. 天気予報による天気の予測と実際の電気使用量の関連性を調べる. 天気概況をグラフに示す.品質番号は 8 を最大として利用上注意が必要がどうかを示す値である. 均質番号は番号により観測環境の違いを表している. この値が違う場合には, 同列のデータとして扱うことは難しい. ただし, 本実験で用いる国土交通省が公開しているデータにおいて 8 以外の品質番号を示すデータは存在していない.
電力, 気温, 降水量の 3 要素で相関係数を求めた時, 表のような数値を取る.
それぞれのデータ間においてほとんど相関関係が認められないことがわかった.
天気概況の天気による数値付けは, 用意した天気のデータが 6 時から 18 時までの天気しか用意できないため, 基本的にその日の主体となっている天気を数値にして本実験では扱う.
晴れ関連を「0.0」, 曇り関連を「0.5」, 雨関連を「1.0」として天気に数値をつける.
天気概況においては「時々」, 「一時」, 「後」等の天気の変化が含まれているが, 「時々」は予報期間の半分未満, 「一時」は予報期間の 4 分の 1 未満なので「時々」と「一時」が含まれる場合においては先に述べられている天気が主体であることがわかる.
「後」はその日の天気がどのように変化するかということを表しており, どちらが主な天気か判別することができないため, 「後」前後の天気の数値の平均をその日の天気として扱うこととする.
また, 天気概況において本実験で用いるデータには, 

%%%%%%%%%%%%%%%%%%%%%%%%%%%%%%%%%%%%%%%%%%%%%%%%%%%%%%%%%%%%%%%%%
\begin{table}[!t]
\centering
  \caption{各データ間の相関関係(数値表記)}
  \begin{tabular}{|c|c|c|c|c|} \hline
    ラベル & 電力 & 気温 & 降水量 & 天気 \\ \hline
    電力 & \ & 0.0851 & -0.0237 & -0.0139 \\ \hline
    気温 & 0.0851 & \ & 0.00205 & -0.0213 \\ \hline
    降水量 & -0.0237 & 0.00205 & \ & 0.0399 \\ \hline
    天気 & -0.0139 & -0.0213 & 0.0399 & \ \\ \hline
  \end{tabular}
\end{table}
%%%%%%%%%%%%%%%%%%%%%%%%%%%%%%%%%%%%%%%%%%%%%%%%%%%%%%%%%%%%%%
\begin{figure*}[phb]
\centering
\includegraphics[scale=0.8]{exe_wether.pdf}
\caption{東京都における天気データ例}
\end{figure*}

%\begin{figure}[hb]
%\centering
%\includegraphics[scale=0.5]{wether.png}
% \caption{天気概況用語一覧}
%\end{figure}
%%%%%%%%%%%%%%%%%%%%%%%%%%%%%%%%%%%%%%%%%%%%%%%%%%%%%%%%%%%%%%%
\begin{table*}[t]
  \caption{天気概況用語の説明}
  \begin{tabular}{|c|c|} \hline
    天気概況用語 & 大気の状態 \\ \hline
    「快晴」 & 雲量 1 以下の状態が長く継続している状態 \\ \hline
    「晴」 & 雲量 2 以上 8 以下の状態 \\ \hline
    「曇」 & 雲量 9 以上であり, 中・下層雲量が上層雲量よりも多く, 降水現象がない状態 \\ \hline
    「薄雲」 & 雲量 9 以上であり, 上層雲量が中・下層雲量よりも多く, 降水現象がない状態 \\ \hline
    「大風」 & 10 分間平均風速が 15.0m/s 以上の風を観測した場合 \\ \hline
    「霧」 & 大気中に浮遊するごく小さな水滴を観測し, 水平視程が 1km 未満の場合 \\ \hline
    「霧雨」 & きわめて多数の細かい水滴だけがかなり一様に降る降水を観測した場合 \\ \hline
    「雨」 & 雨を観測した場合 \\ \hline
    「大雨」 & 「雨」の場合で, 特に降水量が 30.0mm 以上の状態 \\ \hline
    「暴風雨」 & 「大雨」かつ「大風」を観測した場合 \\ \hline
    「みぞれ」 & 雨と雪が混在して降る降水を観測した場合 \\ \hline
    「雪」 & 雪を観測した場合 \\ \hline
    「大雪」 &
    \begin{tabular}{c}
    「雪」の場合で, 北海道, 青森, 秋田, 盛岡, 山形, 新潟, 金沢, 富山, 長野, 福井, 松江においては\\当該時間帯の降雪の深さが 20cm 以上であった場合. \\また, それ以外の地域においては降雪の深さが 10cm 以上であった場合
    \end{tabular}\\ \hline
    「暴風雪」 & 「大雪」かつ「大風を」を観測した場合 \\ \hline
    「地ふぶき」 & 積もった雪が風のために空中に噴き上げられ, それにより視程が 1km 未満の状態 \\ \hline
    「ふぶき」 & 「地ふぶき」かつ「雪」を観測した場合 \\ \hline
    「ひょう」 & 直径 5mm 以上の氷の粒またはかたまりの降水を観測した場合 \\ \hline
    「あられ」 & 直径が概ね 5mm 未満の白色不透明・半透明または透明な氷の粒の降水を観測した場合 \\ \hline
    「雷」 & 雷電または強度 1 以上の雷鳴のいずれかを観測した場合 \\ \hline
    「×」 & 何らかの理由(煙霧による視程障害等)により, 雲の状態が不明で, 天気概況が不明または欠測である場合 \\ \hline
  \end{tabular}
\end{table*}
%%%%%%%%%%%%%%%%%%%%%%%%%%%%%%%%%%%%%%%%%%%%%%%%%%%%%%%%%%%%%%

%\section{先週までの作業}
%\begin{itemize}
%
%\end{itemize}

\section{今週の作業}
\begin{itemize}

\begin{figure*}[phb]
\centering
\includegraphics[scale=0.8]{replace_1.png}
\caption{天気概況変更前}
\end{figure*}
\begin{figure*}[phb]
\centering
\includegraphics[scale=0.8]{replace_2.png}
\caption{天気概況変更後(途中)}
\end{figure*}
%%%%%%%%%%%%%%%%%%%%%%%%%%%%%%%%%%%%%%%%%%%%%%%%%%%%%%%%%%%%%%%%%
\begin{table*}[t]
\centering
  \caption{天気概況による数値付け}
  \begin{tabular}{|c|c|} \hline
    天気概況 & 対応する数値 \\ \hline
    「快晴」 & 0.0 \\ \hline
    「晴」 & 0.0 \\ \hline
    「曇」 & 0.5 \\ \hline
    「薄雲」 & 0.5 \\ \hline
    「大風」 & 0.3 \\ \hline
    「霧」 & 0.6 \\ \hline
    「霧雨」 & 0.7 \\ \hline
    「雨」 & 1.0 \\ \hline
    「大雨」 & 1.0 \\ \hline
    「暴風雨」 & 1.0 \\ \hline
    「みぞれ」 & 1.0 \\ \hline
    「雪」 & 1.0 \\ \hline
    「大雪」 & 1.0 \\ \hline
    「暴風雪」 & 1.0 \\ \hline
    「地ふぶき」 & 1.0 \\ \hline
    「ふぶき」 & 1.0 \\ \hline
    「ひょう」 & 1.0 \\ \hline
    「あられ」 & 1.0 \\ \hline
    「雷」 & 1.0 \\ \hline
    「×」 & 0.0 \\ \hline
  \end{tabular}
\end{table*}
        \item 天気にも数値をつけて, 4 要素での相関をとったが全然相関がなかった. それぞれを 24 個ずつ足して行って 1 日のデータにして相関を見る.
        \item 1 日のデータにして予測を行う.
%        \item 6 時から 18 時のデータを用いて, 晴れ関連を「0.0」, 雨関連を「1.0」として天気に数値をつける. 後や一時, 後一時, 時々などをどう扱うか決める. これらの要素が細かすぎて細分化が難しいため, 1 日のなかで 比率が一番大きいものを主体として数値を与えたい. 例えば時々では, 後ろにくるほうが 2 分の 1 以下なので後半部は消す.
        \item 

\begin{figure*}[phb]
\centering
\includegraphics[scale=0.5]{kekka.png}
\caption{RNN の結果例}
\end{figure*}
\end{itemize}

%\section{来週以降の作業}
%\begin{itemize}
%\end{itemize}
\section{地方会発表}
地方会で発表する内容\\
タイトル\\
深層学習を用いた電気使用量予測\\
1.はじめに\\
研究背景, 研究概要\\
2.用いるモデル構造\\
RNN, LSTM\\
3.用いるデータ\\
それぞれのデータについての説明, 相関関係\\
4.結果\\
5.考察\\

\section{紀要内容}
紀要の内容\\
1 はじめに\\
1.1 研究背景\\
1.2 研究目的\\
2 用いる深層学習モデル\\
2.1 RNN\\
2.2 LSTM\\
3 実装方法\\
4 用いるデータ\\
4.1 電力使用量\\
4.2 気温\\
4.3 降水量\\
4.4 天気\\
5 予測結果\\

\section{参考文献}
[1]Rosenblatt, F. (1958). The perceptron: A probabilistic model for information storage and organization in the brain. Psychological Review, 65(6), 386-408.
[1]浅川伸一. python で体験する深層学習. コロナ社, 2016.
[2]William Lotter, Gabriel Kreiman, David Cox, “Deep Predictive Coding Networks for Video Prediction and Unsupervised Learning”, ICLR, 2017
[3]ニューラルネットによる人の基本表情認識 


\end{document}
