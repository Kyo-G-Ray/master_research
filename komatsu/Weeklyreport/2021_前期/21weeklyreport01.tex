\documentclass[a4j,twocolumn]{ltjarticle}

%ページ設定
\setlength{\topmargin}{-10mm}
\setlength{\oddsidemargin}{-5mm}
\setlength{\evensidemargin}{-5mm}

\setlength{\textheight}{230mm}
\setlength{\textwidth}{170mm}
\setlength{\columnsep}{10mm}%コラムとコラムの間.よってコラムの横幅は(texwidth-columnsep)/2
\setlength{\footskip}{15mm}
%パッケージ
\usepackage{graphicx,color}
\usepackage{bm}
\usepackage{amsmath}
\usepackage{amsfonts}

%数学関数
\newcommand{\B}[1]{\bm #1} 
\newcommand{\s}[2]{{#1}\cdot{#2}} 
\newcommand{\R}{\mathbb{R}}
\newcommand{\M}{\mathbb{M}}
\newcommand{\df}[2]{\displaystyle{\frac{#1}{#2}}}
\newcommand{\qed}{
\begin{flushright}
\vspace{-8mm}
$\Box$
\end{flushright}
}

%ミニセクション
\newcommand{\minisection}[2]{\flushleft{\bf (#1)#2}\\ \ \ \ }

%図,表の名前変更
\renewcommand{\figurename}{Fig.}
\renewcommand{\tablename}{Tab.}

% refの拡張
\newcommand{\reffig}[1]{\figurename \ref{#1}}
\newcommand{\refeq}[1]{式(\ref{#1})}
\newcommand{\reftab}[1]{\tablename \ref{#1}}

% 図
\newcommand{\includefigure}[3]{
\begin{center}
\includegraphics[width=80mm]{#1}
\caption{#2}
\label{#3}
\end{center}
}
% 図 widthを変更したいときはこっち
\newcommand{\includefigurewidth}[4]{
\begin{center}
\includegraphics[width=#2]{#1}
\caption{#3}
\label{#4}
\end{center}
}

\usepackage{setspace} % setspaceパッケージのインクルード
\usepackage{enumitem}
\usepackage{graphicx}
\usepackage{amsmath}
%「Weekly Report」 
\newcommand{\Weekly}[5]{
\twocolumn[
 \begin{center}
  \bf
 第 #1 回 Weekly Report\\
 \huge
深層学習を用いた東京都における電気代予測\\

 \end{center}
 \begin{flushright}
  #2 月\ \ \  #3 日 \ \ \ #4 \\\
  #5
 \end{flushright}
]
}
%\setstretch{0.5} % ページ全体の行間を設定

\begin{document}

\Weekly{1}{4}{13}{(火)}{\ 小松 大起}
\section{はじめに}
\subsection{研究背景}
近年, 情報処理技術として知的処理技術の一つである深層学習が様々な分野で用いられている. 深層学習とは, ニューロンの層が多段に組み上げられたニューラルネットワークのことを指す.[1]ニューラルネットワークとは人間の脳の仕組みから着想を得たものであり, 神経回路網をコンピュータ上で表現しようと作られた数理的モデルである.深層学習で用いられる分野としては株価予想や人物認識や表情認識, 擬似的なデータを生成するアルゴリズムである GAN を用いた画像生成などに挙げられる画像処理, 話し言葉や書き言葉など我々が普段使うような自然言語を対象として, それらの言葉が持つ意味を解析する自然言語処理などがある.

\subsection{研究目的}

人間の認知は時間経過による視覚世界の変化の予測が可能である. 近年では実際に予測動画を作る研究も行われてきている.[2]電気料金は
従来の深層学習においては静止画の画像処理が中心であったが, 本研究では動画像から未来フレームの画像生成へと拡張を行い, 危険認知や行動予測などへの予測画像応用を行うための予測画像生成及び, その生成画像の評価方法を明らかにすることである.
\section{RNN}
\section{LSTM}

\section{活性化関数}
活性化関数とは, ニューロン間の移動に伴い入力値を別の数値に変換して出力するための関数のことである.
\subsection{ステップ関数}
ステップ関数は, 入力が 0 未満の場合には常に出力値が 0 となり, 0 以上の場合には常に出力値が 1 となるような関数を指す. ステップ関数は, パーセプトロンから用いられている関数であり入力 0 を起点として階段状のグラフを示す. この起点を閾値と呼ぶ. 入力を $x$ として $f(x)$ を出力とすると数式は以下の式で表される.
\begin{equation}
f(x)= \begin{cases}
0, & (x < 0)\\
1, & (x \geq 1)
\end{cases}
\end{equation}

%\section{先週までの作業}
%\begin{itemize}
%        \item ES 書くの難しかった.
%\end{itemize}

\section{今週の作業}
\begin{itemize}
        \item 金曜日面接
        \item 
\end{itemize}

\section{来週以降の作業}
\begin{itemize}
         \item オートエンコーダで中間層を扱えるようにする
\end{itemize}

\section{参考文献}
[1]浅川伸一. python で体験する深層学習. コロナ社, 2016.
[2]William Lotter, Gabriel Kreiman, David Cox, “Deep Predictive Coding Networks for Video Prediction and Unsupervised Learning”, ICLR, 2017



\end{document}
