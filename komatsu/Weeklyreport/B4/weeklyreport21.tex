\documentclass[a4j,twocolumn]{ltjarticle}

%ページ設定
\setlength{\topmargin}{-10mm}
\setlength{\oddsidemargin}{-5mm}
\setlength{\evensidemargin}{-5mm}

\setlength{\textheight}{230mm}
\setlength{\textwidth}{170mm}
\setlength{\columnsep}{10mm}%コラムとコラムの間.よってコラムの横幅は(texwidth-columnsep)/2
\setlength{\footskip}{15mm}
%パッケージ
\usepackage{graphicx,color}
\usepackage{bm}
\usepackage{amsmath}
\usepackage{amsfonts}

%数学関数
\newcommand{\B}[1]{\bm #1} 
\newcommand{\s}[2]{{#1}\cdot{#2}} 
\newcommand{\R}{\mathbb{R}}
\newcommand{\M}{\mathbb{M}}
\newcommand{\df}[2]{\displaystyle{\frac{#1}{#2}}}
\newcommand{\qed}{
\begin{flushright}
\vspace{-8mm}
$\Box$
\end{flushright}
}

%ミニセクション
\newcommand{\minisection}[2]{\flushleft{\bf (#1)#2}\\ \ \ \ }

%図,表の名前変更
\renewcommand{\figurename}{Fig.}
\renewcommand{\tablename}{Tab.}

% refの拡張
\newcommand{\reffig}[1]{\figurename \ref{#1}}
\newcommand{\refeq}[1]{式(\ref{#1})}
\newcommand{\reftab}[1]{\tablename \ref{#1}}

% 図
\newcommand{\includefigure}[3]{
\begin{center}
\includegraphics[width=80mm]{#1}
\caption{#2}
\label{#3}
\end{center}
}
% 図 widthを変更したいときはこっち
\newcommand{\includefigurewidth}[4]{
\begin{center}
\includegraphics[width=#2]{#1}
\caption{#3}
\label{#4}
\end{center}
}

\usepackage{setspace} % setspaceパッケージのインクルード
\usepackage{enumitem}

%「Weekly Report」 
\newcommand{\Weekly}[5]{
\twocolumn[
 \begin{center}
  \bf
 第 #1 回 Weekly Report\\
 \huge
深層学習による動画像からの表情認識手法の開発\\

 \end{center}
 \begin{flushright}
  #2 月\ \ \  #3 日 \ \ \ #4 \\\
  #5
 \end{flushright}
]
}
%\setstretch{0.5} % ページ全体の行間を設定

\begin{document}

\Weekly{21}{10}{30}{(水)}{\ 小松 大起}

\section{動画像による表情認識を行う深層学習モデル}
%
%%%%%%%%%%%%%%%%%%%%%%%%%%%%%%%%%%%%%%%%%%%%%%%%%%%%%%%%%%%%%%%
\vspace{-3mm}
\subsection{CNN}
\vspace{-2mm}
CNN は,
畳み込み層とプーリング層を 1 つのペアーとし,
それらが複数回重ね合せて構成される順方向性ニューラルネットワークである.
ここで,対象とする画像を $X \times Y$ pixels の RGB の階調値とし,
$k$ 番目の階調の素子 $(i,\ j)$ の画素値を $I^{(k)}_{ij}$ とする.
ただし,$k = 1$ が R,$k = 2$ が G,$k = 3$ が B とする.
最初の層の畳み込み層の $a$ 番目のフィルターの
$(i,\ j)$ 番目の素子の内部状態を
$y^{(1)(a)}_{ij}$,
その出力を
$\tilde{y}_{ij}^{(1)(a)}$,
プーリング層の出力を
$z^{(1)(a)}_{ij}$
とすると,
各々以下のように与えられる.
%%%%%%%%%%%%%%%%
\vspace{-2mm}
\begin{equation}
\footnotesize
  y^{(1) (a)}_{ij} = \sum_{k=1}^3
  \left(\sum_{x \in W} \sum_{y \in W}
         w^{(1)(a)(k)}_{ij} I^{(k)}_{i + x,\ j + y}  + b^{(1)(a)(k)}_{ij}
         \right)
  %%%%%
  \label{eq:convolution1st}
\end{equation}
\vspace{-2mm}
%%%%%%%%%%%%%%%%
\begin{equation}
  \footnotesize
  \tilde{y}_{ij}^{(1)(a)} = \max \left(y^{(1) (a)}_{ij},\ 0 \right)
  %%%%%
  \label{eq:OutputConvolution}
\end{equation}
\vspace{-2mm}
%%%%%%%%%%%%%%%
\begin{equation}
  \footnotesize
  z_{ij}^{(1)(a)} = \max_{x \in W,\ y\in W} \tilde{y}_{i + x,\ j + y}^{(1)(a)}
  %%%%%
  \label{eq:Pooling}
\end{equation}
%\vspace{-2mm}
%%%%%%%%%%%%%%%%
\noindent 
ここで,
$w^{(1)(a)(k)}_{ij}$ は入力層と畳み込み層間のシナプス結合加重,
$W$ は各素子が入力を受ける範囲を与える配置集合(受容野),
$b^{(1)(a)(k)}_{ij}$ は閾値である.

$\ell$ 番目の層の畳み込み層の出力 $\tilde{y}_{ij}^{(\ell)(a)}$
及びプーリング層の出力
$z^{(\ell)(a)}_{ij}$
は式(\ref{eq:OutputConvolution})及び(\ref{eq:Pooling})
と同じであるが,
$\ell$ 番目の層の畳み込み層の内部状態 $y^{(\ell)(a)}_{ij}$ は異なり,
以下の式で与えられる.
%%%%%%%%%%%%%%%%
\vspace{-2mm}
\begin{equation}
\footnotesize
  y^{(\ell) (a)}_{ij} = \sum_{\alpha=1}^{N(\ell - 1)} \sum_{x \in W} \sum_{y \in W}
     w^{(\ell)(a,\ \alpha)}_{ij} z^{(\ell - 1)(\alpha)}_{i + x,\ j + y}
         + b^{(\ell)(a)}_{ij}
  %%%%%
  \label{eq:convolutionLth}
\end{equation}
\vspace{-2mm}
%%%%%%%%%%%%%%%%

最終層($L$)の内部状態 $y^{(L)}_{k}$ は,
前層のプーリング層の出力 $z^{(L - 1)(a)}_{ij}$ との全結合として,
以下のように与えられる.
%%%%%%%%%%%%%%%%%%%%%%
\vspace{-2mm}
\begin{equation}
  \footnotesize
  y^{(L)}_{k} = \sum_{\alpha=1}^{N(L - 1)} \sum_{i} \sum_{j}
     w^{(L)(\alpha)}_{kij} z^{(L - 1)(\alpha)}_{ij}
         + b^{(L)_k}
   %%%%%
  \label{eq:OutputInternal}
\end{equation}
%%%%%%%%%%%%%%%%%%%%%%%
そして,その出力は,ソフトマックス関数により,
以下のように与えられる.
%%%%%%%%%%%%%%%%%%%%%%
\vspace{-2mm}
\begin{equation}
  \footnotesize
\tilde{y}_{k}^{(L)} = \frac{y^{(L)}_{k}}{\sum_i y^{(L)}_{i}}
   %%%%%
  \label{eq:Output}
\end{equation}
%%%%%%%%%%%%%%%%%%%%%%

%%%%%%%%%%%%%%%%%%%%%%%%%%%%%%%%%%%%%%%%%%%%%
%
\vspace{-9mm}
\subsection{Long Short-Term Memory(LSTM)}
\vspace{-3mm}
%
%%%%%%%%%%%%%%%%%%%%%%%%%%%%%%%%%%%%%%%%%%%%%
LSTM は RNN を拡張したものであり,
畳み込み層の内部状態を,
以下のように変更した.
%%%%%%%%%%%%%%%%%%%%%%%%%
\vspace{-2mm}
\begin{equation}
\tiny
  y^{(\ell) (a)}_{ij}(t) = y^{(\ell) (a)}_{ij} +
  \sum_{\tau = 1}^T \sum_{\alpha=1}^{N(\ell)} \sum_{x \in W} \sum_{y \in W}
  v^{(\ell)(a,\ \alpha)}_{ij}
  y^{(\ell)(\alpha)}_{i + x,\ j + y}(t - \tau)
  %%
  \label{eq:LSTM}
\end{equation}
\vspace{-1mm}
%%%%%%%%%%%%%%%%%
\noindent 
ここで,$y^{(\ell) (a)}_{ij}$ は式(\ref{eq:convolutionLth}) であり,
$ y^{(\ell) (a)}_{ij}(t)$ は $t$ 回目の学習時の畳み込み層の内部状態の
値である.
RNN は,LSTM の $T = 1$ に相当する.
数値実験では,$T = 10$ としている.
%%%%%%%%%%%%%%%%%%%%%%%%%%%%%%%%%%%%%%%%%%%
%
\section{先週までの作業}
\begin{itemize}
\item 院試が終わった
\end{itemize}

\section{今週の作業}
\begin{itemize}
\item 表情画像を集める
 \item 卒論を書き始める
\end{itemize}

\section{来週以降の作業}
\begin{itemize}
\item 実験を行う
  \item 卒論
\end{itemize}


\section{章立て}
\begin{itemize}
\item 1 はじめに\\
  \ 1.1 研究背景\\
  \ 1.2 研究の目的\\
  2 ニューラルネットワーク\\
  \ 2.1 畳み込みニューラルネットワークの概要\\
  \ \ 2.1.1 畳込み層\\
  \ \ 2.1.2 プーリング層\\
  \ \ 2.1.3 全結合層\\
  \ \ 2.1.4 CNN の構成と動作方程式\\
  \ \ 2.1.5 CNN の学習方法\\
  \ 2.2 RNN ( Recurrent Neural Network )\\
  \ 2.3 LSTM ( Long Short - Term Memory)\\
  \ 2.4 CNN を用いた表情認識実験\\
  \ \ 2.4.1 CNN の構造\\
  \ \ 2.4.2 表情認識に用いる表情画像\\
  \ \ 2.4.3 実験方法\\
  \ \ 2.4.4 結果\\
  \ \ 2.4.5 考察\\
  3 動画像を用いた表情認識\\
  \ 3.1 CNN\\
  \ \ 3.1.1 畳込み層\\
  \ \ 3.1.2 プーリング層\\
  \ \ 3.1.3 全結合層\\
  \ 3.2 RNN\\
  \ 3.3 LSTM\\
  \ 3.4 学習手法\\
  \ 3.5 実験に用いる表情画像\\
  \ 3.6 表情認識結果\\
  4 考察\\
  5 まとめ\\
  \ 5.1 今後の課題\\
\end{itemize}
\end{document}
