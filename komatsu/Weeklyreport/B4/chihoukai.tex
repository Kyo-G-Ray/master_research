% ==============================================
% 2019年度電気・情報関係学会北陸支部連合大会
% 講演論文投稿用 pLaTeX2e 版テンプレートファイル(2段組)
% Ver. 1.1: 2006 May 19
%   途中不明
% Ver. 2011.06a
% Ver. 2012.05: 年号等の変更
% Ver. 2013: utf8
% Usage: platex jhes2019-platex-template-utf8.tex
% ==============================================
\documentclass[a4paper,10pt]{jarticle}
%\usepackage[dvips]{graphicx}
%\usepackage{amsfonts,latexsym}

% ==============================================
% page format
%  textwidth  = paperwidth(210mm) -oddsidemargin(13mm) -evensidemargin(13mm)
%  textheight = paperheight(297mm) -topmargin(20mm-header分) -bottommargin(18mm)
% ==============================================
  \setlength{\paperwidth}{210mm}
  \setlength{\textwidth}{\paperwidth}
  \addtolength{\textwidth}{-13mm} % oddside margin
  \addtolength{\textwidth}{-13mm} % evenside margin
  \setlength{\oddsidemargin}{-1in}
  \addtolength{\oddsidemargin}{13mm} %%%% ここで左右を調節
%
  \setlength{\paperheight}{297mm}
  \setlength{\textheight}{\paperheight}
  \addtolength{\textheight}{-28mm} % 文章領域の高さの調整
  \setlength{\topmargin}{-15mm}    % ここで上下位置を調節
%
  \setlength{\headheight}{0in}
  \setlength{\headsep}{0in}
  \setlength{\columnsep}{10mm}

\makeatletter
\renewcommand{\section}{\@startsection{section}{1}{\z@}%
   {1.5\Cvs \@plus.5\Cvs \@minus.2\Cvs}%
   {.5\Cvs \@plus.3\Cvs}%
   {\reset@font\large\bfseries}}   %section見出しの文字サイズをlargeに変更
\renewcommand{\subsection}{\@startsection{subsection}{2}{\z@}%
   {1.5\Cvs \@plus.5\Cvs \@minus.2\Cvs}%
   {.5\Cvs \@plus.3\Cvs}%
   {\reset@font\normalsize\bfseries}} %subsectoin見出しの文字サイズをnormalsizeに変更
\makeatother

  \pagestyle{empty}

\begin{document}
\twocolumn[% -- 1段組にしたい場合は,ここをコメントアウトする
% ==============================================
% ページヘッダ(変更禁止)
% ==============================================
%
\begin{center}
{\Large 2019年度電気・情報関係学会北陸支部連合大会}\\
\vspace{-2mm}\rule{\textwidth}{1mm}
\end{center}
% ==============================================
% 原稿はここから
% ==============================================
%
% 講演題目 =====================================
\hspace*{33mm} 
\begin{minipage}{118mm}  %210mm-13mm-13mm-33mm-33mm
\begin{center}    %左づめの場合はflushleft
{\Large \bf
ネットワークの構造の違いによる表情認識性能の差異
%\\ \vspace*{4mm}	% 複数行の場合,強制改行して,行間を調整
%複数行になるようならここに書く
}
\end{center}
\end{minipage}
%

\vspace*{4mm}	% ここで行間を調整
%
% 著者名(所属) =====================================
\hspace*{33.0mm} %講演番号用空欄
\begin{minipage}{118mm}  %210mm-13mm-13mm-33mm-33mm
\begin{center}    %左づめの場合はflushleft
{\large
{小松 大起(福井大学)}・{岡田 直人(所属)}
}
\end{center}
\end{minipage}

\vspace*{6mm}	% ここで行間を調整
] % -- 1段組にしたい場合は,ここをコメントアウトする


% 本文 =====================================

%\noindent

% section 1 ----
\section{はじめに}

%このファイル jhes2019-platex-template-utf8.tex は,2019年度電気・情報関係学会北陸支部連合大会の
%講演論文投稿用の p\LaTeX2e 版テンプレートファイルで,文字コードは \mbox{UTF-8} で記述されています.
%本文は2段組としていますが,1段組みでもかまいません.

%このファイルを使用すると,
%講演題目は 14pt のボールド・ゴシック体,
%著者名(所属)は 12pt のローマン・明朝体,本文は 10pt のローマン・明朝体,
%本文見出しは 12pt のボールド・ゴシック体になります.

本研究では,ネットワークの構造の違いによる表情認識の差異を調べることを目的とする.

先行研究では CNN,RNN,LSTM による表情認識性能の差異を調べたが,問題点として挙げられるのは1つめはデータの不足,不備による識別率の低下やそもそもの学習が行えないことである.また,学習が行えた場合でも学習後に期待される汎化性が得られないことがある.この問題点に対する解決法はデータを増やすことによって検査データや未学習のデータに対する識別率が改善させることであると考える.しかし,得られたデータに対し,そのデータの有用性を考慮した上で取捨選択の必要がある.

2つ目は,CNN,RNN,LSTM による識別率において LSTM の識別率が高いことからもネットワーク構造によって変わってしまうことである.CNN,RNN,CNN と RNN の組み合わせにおいてのネットワーク構造の違いに差異が出るのかどうかを調べることがこの研究の目的である.

% section 2 ----
\section{動画像による表情認識を行う深層学習モデル}

\begin{enumerate}
\item CNNについて:\\

コンボリューションニューラルネットワークは,畳み込み層とプーリング層を持つネットワークモデルで源流となるネオコグニトロンでも S 層(単純細胞層)と C 層(複雑細胞層)を組み合わせたネットワークモデルとなっている.S 層,C 層はそれぞれ局所的な受容野を持ち,S 層は前層の出力の特徴を抽出する特徴検出器として働き,C 層はそれらの S 層の複数の素子から出力を受ける.C 層の素子は S 層よりも少なく設定されるため S 層の出力を荒くサンプリングするような働きとなっている.また C 層の出力は S 層の出力に対し位置変動の微小変化を許容する能力を有している.ネオコグニトロンではこれらの層によって層が下層になるほど画像に対して,大域的で,複雑な特徴の抽出を行うようになっている.CNN モデルにも同様の能力が備わっており,画像認識や手書き文字認識において位置ずれや変形に対する正しい識別が重要な能力の1つである.

\item RNNについて:\\
%上20mm,下18mm,右13mm,左13mmです.
  %ヘッダー部は変更しないでください.ページ番号は入れないでください.

画像などのデータは1つのベクトルを1つの入力として扱うが,時系列データはいくつかのデータが入力データ群となり,このデータ群を複数扱う必要がある.時系列データの例としては気象データや株価の推移などが挙げられる.これらのデータ群には普通のデータの集まりと違う点として,データの並びに意味を持つことが挙げられる.これらのデータを扱うニューラルネットワークとしてリカレントニューラルネットワークが開発された.

リカレントニューラルネットワークでは,こうしたデータの並びに規則性・パターンがある,もしくはありそうなデータを学習することで未知の時系列データが与えられたときに,そのデータの未来の状態を予測する.

\item CNNとRNNの組み合わせについて:\\
  
これら2つの機能を合わせたものを組み合わせとして扱う.

%講演題目,著者名(所属)および本文は,黒字を使用してください.
%\item 講演題目,著者名(所属):\\
%講演題目は文章領域の最上部に,その下に著者名(所属)を書いてください.講演者に○印等を付けないでください.
%\item 図,表:\\
%図,表およびその中の文字は,印刷時に判読できる大きさとし,小さくならないように注意してください.グラフ,画像,写真などの図は,明瞭な色のカラーでも構いません.
%\item 言語:\\
%日本語または英語で作成してください.
\end{enumerate}

% section 3 ----
\section{投稿PDFファイル}
投稿ファイルはPDF(拡張子.pdf)のみです.
Adobe Reader最新版で正常に表示および印刷できることを確認してください.
\begin{enumerate}
\item ファイル形式:\\
セキュリティ設定をしないで,PDFファイルを作成してください.
写真や画像を含む場合,画質の劣化に注意してください.
\item ファイルサイズ:\\
2MB以内としてください.
\item フォント埋め込み:\\
文字化けを防ぐために,フォントを埋め込んでください.
\end{enumerate}

% section 4 ----
\section*{問い合せ先・事務局}
\noindent
%〒929-0392 石川県河北郡津幡町北中条\\

%石川工業高等専門学校\\
%2019年度電気・情報関係学会北陸支部連合大会 事務局 \\
%E-mail: jhes2019@ishikawa-nct.ac.jp\\
%Tel: 076-288-8124 (事務局 山田・上町) \\
%大会Webサイト:http://2019.jhes.jp

% ==============================================
% 原稿はここまで
% ==============================================
\end{document}
