\documentclass[a4j,twocolumn]{ltjarticle}

%ページ設定
\setlength{\topmargin}{-10mm}
\setlength{\oddsidemargin}{-5mm}
\setlength{\evensidemargin}{-5mm}

\setlength{\textheight}{230mm}
\setlength{\textwidth}{170mm}
\setlength{\columnsep}{10mm}%コラムとコラムの間.よってコラムの横幅は(texwidth-columnsep)/2
\setlength{\footskip}{15mm}
%パッケージ
\usepackage{graphicx,color}
\usepackage{bm}
\usepackage{amsmath}
\usepackage{amsfonts}

%数学関数
\newcommand{\B}[1]{\bm #1} 
\newcommand{\s}[2]{{#1}\cdot{#2}} 
\newcommand{\R}{\mathbb{R}}
\newcommand{\M}{\mathbb{M}}
\newcommand{\df}[2]{\displaystyle{\frac{#1}{#2}}}
\newcommand{\qed}{
\begin{flushright}
\vspace{-8mm}
$\Box$
\end{flushright}
}

%ミニセクション
\newcommand{\minisection}[2]{\flushleft{\bf (#1)#2}\\ \ \ \ }

%図,表の名前変更
\renewcommand{\figurename}{Fig.}
\renewcommand{\tablename}{Tab.}

% refの拡張
\newcommand{\reffig}[1]{\figurename \ref{#1}}
\newcommand{\refeq}[1]{式(\ref{#1})}
\newcommand{\reftab}[1]{\tablename \ref{#1}}

% 図
\newcommand{\includefigure}[3]{
\begin{center}
\includegraphics[width=80mm]{#1}
\caption{#2}
\label{#3}
\end{center}
}
% 図 widthを変更したいときはこっち
\newcommand{\includefigurewidth}[4]{
\begin{center}
\includegraphics[width=#2]{#1}
\caption{#3}
\label{#4}
\end{center}
}

\usepackage{setspace} % setspaceパッケージのインクルード
\usepackage{enumitem}

%「Weekly Report」 
\newcommand{\Weekly}[5]{
\twocolumn[
 \begin{center}
  \bf
 第 #1 回 Weekly Report\\
 \huge
深層学習による動画像からの表情認識手法の開発\\

 \end{center}
 \begin{flushright}
  #2 月\ \ \  #3 日 \ \ \ #4 \\\
  #5
 \end{flushright}
]
}
%\setstretch{0.5} % ページ全体の行間を設定

\begin{document}

\Weekly{10}{6}{19}{(水)}{\ 小松 大起}

\section{CNNとは}
\begin{itemize}
\item Convolutional Neural Networkの略称であり、ニューラルネットワークに畳み込みという操作を加えたものである。 その畳み込みとは、画像処理でよく利用される手法で、カーネルと呼ばれる格子状の数値データと、カーネルと同サイズの部分画像の数値データについて、要素ごとの積の和を計算することで、1つの数値に変換する処理のことである。この変換処理を、ウィンドウを少しずつずらして処理を行うことで、小さい格子状の数値データに変換することができる。

  \section{地方会で発表したいこと}
  \begin{itemize}
  \item 動画での表情認識を行い、その結果をもとにしてどのようにしたら動画での表情認識で精度を挙げられるかの考察を行う。
    
\section{先週までの作業}
\begin{itemize}
\item deianをtestingからstretchに変更した。        
\end{itemize}

\section{今週の作業}
\begin{itemize}
\item pyファイルを実行するたびにいろいろなライブラリのバージョンの整合性が合わずエラーが起きてしまう。どうしても使えなさそうなものは必要かどうかを判断しつつ使わないという判断も行っていきたい。
\item 
\end{itemize}

\section{来週以降の作業}
\begin{itemize}
          \item kerasでCNNを実現し、静止画での表情認識を行う。
          \item 表情の動画像のデータをもっと集める。
\end{itemize}




\end{document}
