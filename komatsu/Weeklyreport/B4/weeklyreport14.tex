\documentclass[a4j,twocolumn]{ltjarticle}

%ページ設定
\setlength{\topmargin}{-10mm}
\setlength{\oddsidemargin}{-5mm}
\setlength{\evensidemargin}{-5mm}

\setlength{\textheight}{230mm}
\setlength{\textwidth}{170mm}
\setlength{\columnsep}{10mm}%コラムとコラムの間.よってコラムの横幅は(texwidth-columnsep)/2
\setlength{\footskip}{15mm}
%パッケージ
\usepackage{graphicx,color}
\usepackage{bm}
\usepackage{amsmath}
\usepackage{amsfonts}

%数学関数
\newcommand{\B}[1]{\bm #1} 
\newcommand{\s}[2]{{#1}\cdot{#2}} 
\newcommand{\R}{\mathbb{R}}
\newcommand{\M}{\mathbb{M}}
\newcommand{\df}[2]{\displaystyle{\frac{#1}{#2}}}
\newcommand{\qed}{
\begin{flushright}
\vspace{-8mm}
$\Box$
\end{flushright}
}

%ミニセクション
\newcommand{\minisection}[2]{\flushleft{\bf (#1)#2}\\ \ \ \ }

%図,表の名前変更
\renewcommand{\figurename}{Fig.}
\renewcommand{\tablename}{Tab.}

% refの拡張
\newcommand{\reffig}[1]{\figurename \ref{#1}}
\newcommand{\refeq}[1]{式(\ref{#1})}
\newcommand{\reftab}[1]{\tablename \ref{#1}}

% 図
\newcommand{\includefigure}[3]{
\begin{center}
\includegraphics[width=80mm]{#1}
\caption{#2}
\label{#3}
\end{center}
}
% 図 widthを変更したいときはこっち
\newcommand{\includefigurewidth}[4]{
\begin{center}
\includegraphics[width=#2]{#1}
\caption{#3}
\label{#4}
\end{center}
}

\usepackage{setspace} % setspaceパッケージのインクルード
\usepackage{enumitem}

%「Weekly Report」 
\newcommand{\Weekly}[5]{
\twocolumn[
 \begin{center}
  \bf
 第 #1 回 Weekly Report\\
 \huge
深層学習による動画像からの表情認識手法の開発\\

 \end{center}
 \begin{flushright}
  #2 月\ \ \  #3 日 \ \ \ #4 \\\
  #5
 \end{flushright}
]
}
%\setstretch{0.5} % ページ全体の行間を設定

\begin{document}

\Weekly{14}{7}{17}{(水)}{\ 小松 大起}

\section{CNNとは}
\begin{itemize}
\item CNN とは多層 NN の一つで、畳込み NN とも呼ばれる.ネットワーク上に畳込み層とプーリング層の2層を交互に持つネットワークである.$対象とする画像を Wpicsel × Wpicsel の RGB の階調値とし,y^{(k)}_{ij}と表す.$ここで k は RGB の色調を表し k = 1,2,3(N = 3)の値をとる.$畳込み層(i,j)の素子の内部状態 x_{ij} は以下で与えられる.$
\begin{equation}
  y^{(a)}_{ij} = \sum_{k=1}^N (\sum_{p=1}^w \sum_{q=1}^w x^{(k)}_{i+p,j+q}h_{ij}^{(k)(a)}+b^{(k)(a)})
\end{equation}
$ここで a はフィルタの数を示し,h_{ijk}^{(k)(a)} は受容野における各素子の係数であり CNN は学習によってこの h_{ijk}^{(k)(a)} を変化させていく,b^{(k)(a)} は色調のしきい値,さらに w × w は受容野の大きさを示し,位置(i,j)の素子は,入力層の限られた範囲からのみ入力を受けることを表している[1].畳込み層の(i,j)の素子の出力および活性化関数f(x)は以下で与えられる.$
\begin{eqnarray}
  \tilde{y}_{ij}^{(a)} = f(y^{(a)}_{ij})\\
  f(x)= max(x,0)
\end{eqnarray}
$であるが多クラス分類のために出力層には活性化関数 g(x)を用いる.
$入力 y_j に対して,最終的な出力 z_j は以下で与えられる.$
\begin{equation}
  z_i = \frac{e^{\tilde{y}_i}}{\sum_{k=1}^n e^{\tilde{y}_k}}
\end{equation}
%で与える.プーリング層の振る舞い方にはいくつかのパターンがあるが今回は最大プーリングと呼ばれる次式を用いる.
%\begin{equation}
 % \tilde{y}_{ij}^{(a)} = \max_{p,q \in P_{ij}}
%\end{equation}

\section{先週までの作業}
\begin{itemize}
\item 地方会の予稿の作成.       
\end{itemize}

\section{今週の作業}
\begin{itemize}
\item 地方会の予稿の作成.
\end{itemize}

\section{来週以降の作業}
\begin{itemize}
          \item 地方会での発表スライドの作成.
          \item CNN についての勉強.
\end{itemize}




\end{document}
