\section{はじめに}

%%%%%%%%%%%%%%%%%%%%%%%%%%%%%%%%%%%%%%%
\subsection{研究背景}
2011 年 3 月 11 日に発生した東日本大震災以降, 低コストかつ効率的な発電方法であり全体の 2 割から 4 割を占めていた原子力発電は安全性に問題があるとされ 2021 年 5 月現在では停止中の 2 基も合わせて合計 9 基の原子力発電所が稼働するのみとなっている[1].
それによって, コストの高い火力発電の量が増え発電コストの増加による電気料金の値上げにつながり, 我々消費者の負担増加に繋がっている[2].
電力発電において電力使用量の予測を行うことにより, 長期的には季節ごとの電力使用量に基づいた最適な時期での化石燃料の調達, 短期的には電力使用量が増える時間帯において各発電の稼働状態を適切に決めることが求められている.
大まかな電力使用量の推移は経験から判別が可能であると思われるが, 深層学習を用いて予測を行うことによってより細かい予測により今後の計画を立てることができるようになる.
適切な予測を行い適切な量の電力供給や燃料調達を行うことができれば電力会社は過剰な発電や, 供給不足による突発的に高コストな発電方法, 効率の悪い発電方法を行う必要がなくなると考えられる.
%近年, 人工知能に関する多くの分野で情報処理技術として知的処理技術の一つである深層学習が用いられている.
%深層学習とは, ニューロンの層が多段に組み上げられたニューラルネットーワークのことを指す.[1]
%深層学習が用いられる分野としては, 人物の行動認識や表情認識に挙げられるような画像処理に関わるものや, 話し言葉や書き言葉などの我々が普段使うような自然言語を対象として, それらの言葉が持つ意味を解析する自然言語処理や, 株価や電気使用量などの予測にも用いられる. 
\newpage
%%%%%%%%%%%%%%%%%%%%%%%%%%%%%%%%%%%%%%%
\subsection{研究目的}
